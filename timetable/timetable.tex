%%%%%%%%%%%%%%%%%%%%%%%%%%%%%%%%%%%%%%%%%
% Conference/Event Timetable
% LaTeX Template
% Version 1.1 (24/9/18)
%
% This template has been downloaded from:
% http://www.LaTeXTemplates.com
%
% Original author:
% Evan Sultanik with modifications by 
% Vel (vel@LaTeXTemplates.com)
%
% License:
% CC BY-NC-SA 3.0 (http://creativecommons.org/licenses/by-nc-sa/3.0/)
%
% Important note:
% This template requires the calendar.sty file to be in the same directory as the
% .tex file. The calendar.sty file provides the necessary structure to create the
% calendar.
%
%%%%%%%%%%%%%%%%%%%%%%%%%%%%%%%%%%%%%%%%%

%----------------------------------------------------------------------------------------
%	PACKAGES AND OTHER DOCUMENT CONFIGURATIONS
%----------------------------------------------------------------------------------------

\documentclass[10pt,letterpaper]{article} % Can also use 11pt for a larger overall font size

\pdfminorversion=5
\pdfobjcompresslevel=3 
\pdfcompresslevel=9

\usepackage{calendar} % Use the calendar.sty style

\usepackage[landscape, letterpaper, margin=1cm]{geometry} % Page dimensions and margins
\usepackage{ifxetex}

\ifxetex
\usepackage{fontspec}
\setmainfont{Raleway}
\fi

\usepackage{datatool}

\DTLloaddb{data}{../data.csv}


\newcommand*{\thevalue}{}
\newcommand*{\getCol}[2]{%
 \DTLgetvalueforkey{\thevalue}{Last}{data}{Talk}{#2}%
}

\newcommand*{\lastname}[2]{
\getCol{1}{#1}
\thevalue 
}


\newcommand*{\thevalues}{}
\newcommand*{\getCols}[2]{%
 \DTLgetvalueforkey{\thevalues}{Time}{data}{Talk}{#2}%
}



\newcommand*{\gettime}[2]{
\getCols{1}{#1}
\kern-1ex\thevalues
}




\begin{document}

\pagestyle{empty} % Disable default headers and footers

\setlength{\parindent}{0pt} % Stop paragraph indentation

\StartingDayNumber=1 % Calendar starting day, default of 1 means Sunday, 2 for Monday, etc

%----------------------------------------------------------------------------------------
%	TITLE SECTION
%----------------------------------------------------------------------------------------

\begin{center}
	\textsc{\LARGE Workshop on Experimental and Computational Fracture Mechanics: \\
	\small Validating peridynamics and phase field models for fracture prediction and 	experimental design}\\ % Title
	\textsc{\large February 2020} % Date
\end{center}

%----------------------------------------------------------------------------------------

\begin{calendar}{\textwidth} % Calendar to be the entire width of the page

\setcounter{calendardate}{23} % Day on which the calendar starts - note that you have to account for blank days

%----------------------------------------------------------------------------------------
%	Monday
%----------------------------------------------------------------------------------------

\day{}{}

%----------------------------------------------------------------------------------------
%	SECOND DAY
%----------------------------------------------------------------------------------------

\day{}{}

%----------------------------------------------------------------------------------------
%	THIRD DAY
%----------------------------------------------------------------------------------------

\day{Arrival}{
\textbf{Reception} The Cook Hotel (7:00-9:00) \\\daysep
}
%----------------------------------------------------------------------------------------
%	FOURTH DAY
%----------------------------------------------------------------------------------------


\day{Day 1}{
\textbf{Registration/Coffee} (8:15-8:45) \\\daysep
Welcoming Remarks (8:45-9:00)\\\daysep
\lastname{1}~ (9:00-10:00) \\\daysep
\textbf{Coffee Break} (10:00-10:30) \\\daysep
\lastname{2}~ (10:30-11:00) \\\daysep
\lastname{3}~ (11:00-11:30) \\\daysep
\lastname{4}~ (11:30-12:00) \\\daysep
\lastname{5}~ (12:00-12:30) \\\daysep
\textbf{Lunch} (12:30-14:00) \\\daysep
\lastname{6}~ (14:00-14:30) \\\daysep
\lastname{7}~ (14:30-15:00) \\\daysep
\lastname{8}~ (15:00-15:30) \\\daysep
\textit{Discussion }  (15:30-16:00) \\\daysep
\textbf{Coffee Break} (16:00-16:30) \\\daysep
\lastname{9}~ (16:30-17:00) \\\daysep
\lastname{10}~ (17:00-17:30) \\\daysep
}

%----------------------------------------------------------------------------------------
%	FIFTH DAY
%----------------------------------------------------------------------------------------

\day{Day 2}{
\textbf{Registration/Coffee} (8:30-9:00) \\\daysep
\lastname{11}~ (9:00-10:00) \\\daysep
\textbf{Coffee Break} (10:00-10:30) \\\daysep
\lastname{12}~ (10:30-11:00) \\\daysep
\lastname{13}~ (11:00-11:30) \\\daysep
\lastname{14}~ (11:30-12:00) \\\daysep
\lastname{15}~ (12:00-12:30) \\\daysep
\textbf{Lunch} (12:30-14:00) \\\daysep
\lastname{16}~ (14:00-14:30) \\\daysep
\lastname{17}~ (14:30-15:00) \\\daysep
\textit{Discussion} (15:00-15:30) \\\daysep
\textbf{Coffee Break}~ (15:30-16:00) \\\daysep
\lastname{18}~ (16:00-17:00) \\\daysep
\lastname{19}~ (17:00-17:30) \\\daysep
\textbf{Banquet} Faculty Club LSU (19:00-21:00)\\
}

%----------------------------------------------------------------------------------------
%	SIXTH DAY
%----------------------------------------------------------------------------------------


\day{Day 3}{
\textbf{Registration/Coffee} (8:30-9:00) \\\daysep
\lastname{20}~ (9:00-09:30) \\\daysep
\lastname{21}~ (9:30-10:00) \\\daysep
\textbf{Coffee Break}~ (10:00-10:30) \\\daysep
\lastname{22}~ (10:30-11:00) \\\daysep
\lastname{23}~ (11:00-11:30) \\\daysep
\lastname{24}~ (11:30-12:00) \\\daysep
General discussion (12:00--12:30) \\\daysep
\textbf{Lunch} (12:30-14:00) \\\daysep
}



%----------------------------------------------------------------------------------------
%	SEVENTH DAY
%----------------------------------------------------------------------------------------





% Note: more days can be added to give the calendar a third or fourth week

%----------------------------------------------------------------------------------------

\finishCalendar
\end{calendar}



\end{document}
