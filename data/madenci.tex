\begin{center}
\textit{Co-Authors: Cagan Diyarogluand and Nam Phan}
\end{center} 
This study presents a new approach to derive the bond-based (BB) Peridynamic (PD) equation of motion   for   elastic   and   isotropic   materials   experiencing   small   deformation.      The   pairwise interaction of the material points lead to bond forces between the two material points in opposite directions with equal magnitude.  The bond force depends on the deformation state between these points.  The original derivation of the BB PD equation of motion by Silling [1] requires calibration in order to determine the micro modulus (bond constant).  Thenew approachemploys the Navier’s displacement equilibrium equations in conjunction with the PD differential operatorby Madenci et al.  [2].    The  resulting  BB  PD  equilibrium  equation  is  not  limited  to  a  specific  value  of  Poisson’s ratio.  It directly invokes the Young’s modulus and Poisson’s ratio with any calibration.  Also, it is capable  of  directly  computing  the stress  and  strain  fields.    The  approach  is  implemented  into  the ANSYS framework by using Matrix 27 elements in a seamless fashion.  Several simulation results establish the fidelity of this approach. \\

\noindent\textbf{References}\\
$[$1$]$ S. A. Silling, Reformulation of Elasticity Theory for Discontinuities and Long-Range Forces,J. Mech. Phys. Solids, 48, 175–209, 2000.\\\newline
$[$2$]$ E. Madenci, A. Barut, M. Dorduncu, Peridynamic differential operator for numerical analysis, Springer, NY, 2019.
