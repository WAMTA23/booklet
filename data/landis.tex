\begin{center}
\textit{Co-Authors: Yu-Sheng Lo, Amin Anvari, K. Ravi-Chandar, Thomas J.R. Hughes, and
Michael J. Borden}
\end{center} 
Over the last few decades, the phase-field approach to fracture has been shown to be a useful tool
for modeling complex crack path evolution. Features including the nucleation, turning, branching,
and merging of cracks as a result of quasi-static mechanical and dynamic loadings are captured
without the need for extra constitutive rules for these phenomena. This presentation will touch on
our recent work on the phase-field modeling approach for fatigue crack growth, R-curve behavior
for brittle fracture in the presence of plastic flow, and modifications for large-scale structures will
be discussed. For fatigue, a modified J-integral will be developed to demonstrate how the phase-
field approach can be used to generate Paris-Law type crack growth rates. A steady-state finite
element method is then applied to generate fits of the phase-field theory to measured crack growth
rate data. Full transient simulations are performed and compared to experimental measurements on
samples where crack turning is induced by the presence of a hole in the vicinity of the crack. To
model R-curve behavior plasticity is introduced into the formulation and adaptive refinement is used
to capture different length scales. Finally, modifications to the damage functions are introduced to
allow for the analysis of large scale structures and some issues are identified and discussed.
