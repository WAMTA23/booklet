The leading-edge of High Performance Computing is challenged by the end of Moore's Law and new applications’ demands in, among other areas, "AI" as the term is being currently employed in the common lexicon. Special Purpose Devices and, in some cases, entire new class of systems have attracted significant investment in recent years with the intent of meeting the rapidly growing demand (at least in appearance) for supervised machine learning platforms. More generally, for both dynamic numeric and informatic problems, the basic data structure may not be sparse matrices but rather time-varying and irregular graphs. AMR and N-body numeric problems and unsupervised machine learning, contextual natural language processing, searches, sorting, hypothesis testing, decision making, and a host of NP complete problems requiring non-deterministic but convergent solutions make up a wide-array of present and future workflows requiring new large-scale solutions. An innovative class of memory-centric architectures are emerging as a research focus to address both dimensions of the design and operation space. One such is the “Active Memory Architecture” under development as an example of a novel memory-centric system incorporating non von Neumann architecture structures, semantic constructs, graph and runtime related overhead primitive mechanisms, and runtime resource management and task scheduling methods. This advanced technical strategy has been sponsored by NASA and is supported by IARPA/ARO. The fundamental principles and planned methods being undertaken with the intent of modeling, simulation, and evaluation will be presented along with a completed FPGA-based graph accelerator prototype.