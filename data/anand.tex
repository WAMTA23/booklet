I will present a gradient-damage theory for fracture of “quasi-brittle” materials under tensile dominated stress states. The theory is developed using the method of virtual-power. The macro- and microforce balances, obtained from the virtual power approach, together with a standard free-energy imbalance equation under isothermal conditions, when supplemented with a set of thermodynamically-consistent constitutive equations provide the governing equations for the theory. The general theory has been specialized to formulate a model for fracture of concrete --- a quasi-brittle material of vast importance. We have numerically implemented our theory in a finite element program, and we present results from representative numerical calculations which show the ability of our simulation capability to reproduce the macroscopic load-deflection characteristics as well as crack-paths during failure of concrete in several technically relevant geometries reported in the literature.\\

\noindent\textbf{References}\\
$[$1$]$ S. Narayan, and L. Anand, A gradient-damage theory for fracture of quasi-brittle materials, Journal of the Mechanics and Physics of Solids, 129, 119-146, 2019.
