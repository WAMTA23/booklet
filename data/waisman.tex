\begin{center}
\textit{Co-Authors: Lampros Svolos1 and Curt A. Bronkhorst}
\end{center} 
Dynamic loading of polycrystalline metallic materials can lead to brittle or ductile fracture de-pending on the loading rates, geometry and material type. Cracks develop rapidly with minimal plasticity and minimal heat dissipation when brittle fracture is observed (e.g. Kalthoff problem under low strain rates). At high strain rates for metallic materials which can accommodate plastic deformation, material instabilities known as shear bands, can occur. Shear bands are narrow localization bands which reduce the stress bearing capacity of the material and act as a precursor to ductile fracture (e.g. cracks that develop rapidly on top of a shear band).A unified model, which has been developed in [1,2], accounts for the two aforementioned failure processes simultaneously. In this model, the phase-field method is used to model crack initiation and propagation, and is coupled to a temperature dependent visco-plastic model that captures shear bands. In this work, an improvement to the unified model is presented in order to capture more accurately the heat transfer across the fracture surfaces. Specifically, an isotropic degradation of the thermal conductivity is proposed, which couples the thermal diffusion process with the extent of damage across a crack. The closed form solution is derived analytically based on a micro-mechanics void extension model of Laplace’s equation. We investigate the behavior of the aforementioned technique on two benchmark problems and show the necessity of suchphysics-based degradation function in dynamic fracture problems.\\

\noindent\textbf{References}\\
$[$1$]$ C. McAuliffe, and H. Waisman, A unified model for metal failure capturing shear banding and fracture,International Journal of Plasticity,65, 131–151, 2015.\\\newline
$[$2$]$ C. McAuliffe, and H. Waisman, A coupled phase field shear band model for ductilebrittle transition in notched plate impacts,Computer Methods in Applied Mechanics and Engineering,305, 173–195, 2016.
