\begin{center}
\textit{Co-Authors: Wayne	Newhauser and Patrick Diehl }
\end{center} 
\begin{center}
Background
\end{center}
One of the most common biological effects of radiation is blood vessel damage, which can lead to deleterious effects such as radiation necrosis or atherosclerotic heart disease. In recent years, several groups have performed computational blood flow simulations in the heart. However, these simulations neglected radiation injury and are limited to the resolution of the imaging modalities, typically around 1-2 millimeters. Other computational methods for modeling radiation dose deposition and biological response had been applied only in small volumes of tissue. To understand the systemic effects of radiation on the entire body, computational methods must surpass greater length and time scales than previously achieved. In this study we aim to simulate radiation damage and analyze the effects of the resulting structural changes on blood flow in a system greater than 34 billion vessels, approximately the size of the human body.

\begin{center}
Methods
\end{center}
Vascular Geometry: The vascular geometry of the system is constructed from a fractal algorithm to generate 3-dimensional scalable vessel networks. The vessels are represented as rigid, cylindrical tubes connected at junctions to form a closed network, with symmetric halves comprising of an arterial tree and a venous tree.
Radiation Transport: Radiation transport is simulated using an amorphous track-structure method to model dose deposition from protons and secondary ionized electrons, or -rays. The biological response of the impacted vessels is modeled to fit experimental data.
Fluid Dynamics: The resulting changes in blood flow are calculated utilizing a special case of the Navier-Stokes equation, known as the Poiseuille equation, for the motion of incompressible Newtonian fluids. The network is cast as a system of linear equations, and iterative Krylov methods are used to solve for the blood flow rates of each vessel.
Computational Aspects: Due to the large-scale nature of the project, we must implement high-performance computing techniques on supercomputer clusters. A single vessel in our system is approximately 68 bytes. Therefore, representing the geometry of an entire human body requires over 1 terabyte of memory, greater than typically available on a single compute node.
To overcome this, we will be integrating the High Performance ParalleX (HPX) C++ library for increased parallelism and concurrency, developed by our collaborators at the STE$\vert\vert$AR group in Louisiana State University’s Center for Computation and Technology (CCT). HPX is an Asynchronous Many Task runtime system that has shown superior parallel efficiency in large-scale projects.

\begin{center}
Preliminary Results
\end{center}
Preliminary results from our laboratory have shown the computational feasibility of calculating blood flow in 17 billion vessels. We have also shown the feasibility of demonstrating whole-organ vascular injury from radiation. This was accomplished on a vascular network the size of the human brain (9 billion vessels) with dose simulated by an amorphous track-structure model consisting of 2 million protons. We are currently generating further preliminary computational results at the CCT’s Rostam supercomputer cluster. 