\begin{center}
\textit{Co-Authors: Tao Ni, Francesco Pesavento, Mirco Zaccariotto, Francesco Scabbia, and Bernhard A Schrefler}
\end{center} 
This paper presents a novel hybrid modeling approach for simulating multi-physics problems involving fracture. The new computational method is applied to hydraulic fracture propagation in saturated porous media: Ordinary State based Peridynamics is used to describe the behavior of the solid phase, including crack propagation, while Classical Continuum Mechanics is used to describe the fluid flow and to evaluate the pore pressure. Classical Biotporoelasticity theory is adopted [1, 2]. The fluid pressure is applied as an internal force to the solid Peridynamic grid, which deforms and may crack under increasing fluid pressure. Crack propagation changes the porosity of the material and the volume in which the fluid is free to move. This information is continuously transferred between the two fields (solid and fluid) to solve the coupled problem. The accuracy of the proposed approach is initially verified by comparing its results with the exact solutions of two examples. Subsequently we will present the solution of several pressure-and fluid-driven crack propagation examples. The phenomenon of fluid pressure oscillation is observed in the fluid-driven crack propagation examples, which is consistent with previously obtained experimental and numerical data [3]. All the presented examples illustrate the capability of the proposed approach to solve problems of hydraulic fracture propagation insaturated porous media. Finally we observe that the solid is completely discretised with a PD based approach, the fluid with a CCM based method and the mentioned ‘coupling’ is a physical coupling involving fluid-structure interaction; that is not to be confused with the numerical coupling [4] often cited in the Peridynamic literature.\\

\noindent\textbf{References}\\
$[$1$]$ Tao Ni; Francesco Pesavento; Mirco Zaccariotto; Ugo Galvanetto; Qizhi Zhu; Bernhard A Schrefler, Hybrid FEM and Peridynamic simulation of hydraulic fracture propagation in saturated porous media, submitted for publication, 2019.\\
$[$2$]$ R. W. Lewis and B. A. Schrefler, The Finite element method in the static and dynamic deformation and consolidation of porous media, John Wiley, 1998.\\
$[$3$]$ T. D. Cao, F. Hussain, B. A. Schrefler, Porous media fracturing dynamics: stepwise crack advancement and fluid pressure oscillations, Journal of the Mechanics and Physics of Solids 111 (2018) 113-133.\\
$[$4$]$ T. Ni, M. Zaccariotto, Q.-Z. Zhu, U. Galvanetto, Coupling of fem and ordinary state-based peridynamics for brittle failure analysis in 3d, Mechanics of Advanced Materials and Structures, (2019) \url{https://doi.org/10.1080/15376494.2019.1602237.}
