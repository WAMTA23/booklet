\begin{center}
\textit{Co-Authors: C.C. Roth, V. Grolleau}
\end{center} 
The comparison of different simple shear experiments revealed that the great potential of the in-plane torsion configuration for characterizing the fracture response of sheet metal. Only the in-plane torsion configuration seems to be able to provide perfectly proportional simple shear loading histories for isotropic materials. State-of-the-art torsion tests are machined free of slits, but a circular groove is introduced to reduce the thickness of the material and to ensure defined strain localization away from the clamped boundaries. However, this configuration tends to limit the observability of the entire shear zone. Here, an enhanced in-plane torsion test is developed using a grooved specimen with full optical access to the specimen surface for DIC measurements and enlarged strain rate range up to several 100/s. Its main feature is a new clamping and loading technique from the inner boundary of the specimen. Validation experiments are performed on specimens extracted fro man aluminum alloy and steel sheets of various ductility. The experimental campaign includes proportional loading, cyclic loading and strain rate jump tests to demonstrate the ability of the newly-proposed technique. Full optical access to the specimen surface is used to reveal the effect of the material anisotropy on the strain field along the sheared area. The findings are compared to results from numerical simulations and with other in-plane shear tests.
