\begin{center}
\textit{Co-Author: Prashant K. Jha}
\end{center} 
We introduce a state based peridynamic model for calculating dynamic fracture. The force interaction is derived from a double well strain energy density function, resulting in a non-monotonic material model. The material properties change in response to evolving internal forces and fracture emerges from the model. The model can be viewed as a regularized fracture model.In the limit of zero nonlocal interaction the model recovers a sharp crack evolution characterized by the classic Griffith free energy of brittle fracture with elastic deformation satisfying the linear elastic wave equation off the crack set, zero traction on crack faces and the kinetic relation between crack tip velocity and crack driving force given in [1], [4], [6], [7], see [3], [5]. We complete the talk with a priori convergence rates for the numerical simulation and several computational examples [2]. This research is funded through ARO Grant W911NF1610456.\\

\noindent\textbf{References}\\
$[$1$]$ L. B. Freund, L. B., 1972. Energy flux into the tip of an extending crack in an elastic solid.J. Elasticity2, 341–349, 1972. \\\newline
$[$2$]$ P. K. Jha and R. P. Lipton. Numerical convergence of finite difference approximations for state based peridynamic fracture models.Comput. Methods. Appl. Mech. Engrg.351,184–225, 2019. \\\newline
$[$3$]$ P. K. Jha and R. P. Lipton. In preparation, 2020.\\\newline
$[$4$]$ B. V. Kostrov and L. V. Nikitin. Some general problems of mechanics of brittle fracture.Arch. Mech. Stosowanej. 22, 749–775, 1970.\\\newline
$[$5$]$ R. P. Lipton and P. K. Jha. Classic dynamic fracture recovered as the limit of a non-local peridynamic model: The single edge notch in tension. ArXiv for Mathematics.ArXiv:1908.07589v4 [math.AP] 8 Nov 2019.\\\newline
$[$6$]$ Y. Slepian. Models and Phenomena in Fracture Mechanics. Foundations of Engineering Mechanics. Springer-Verlag. Berlin, 2002. \\\newline
$[$7$]$ J. R. Willis. Equations of motion for propagating cracks, The mechanics and physics offracture.The Metals Society.57–67, 1975
