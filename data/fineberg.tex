\begin{center}
\textit{Co-Authors: L. Rozen-Levy and J. Kolinski}
\end{center} 
While we have an excellent fundamental understanding of the dynamics of ‘simple’ cracks propagating in brittle solids, we do not fully understand how the path of moving cracks is determined. Here we experimentally study cracks that propagate between 10-95\% of their limiting velocity within a brittle material. We deflect these cracks by either allowing them to interact with sparsely implanted defects or driving them to undergo an intrinsic oscillatory instability in defect-free media. Dense, high-speed measurements of the strain fields surrounding the crack tips obtained via imaging reveal that the paths selected by these rapid and strongly perturbed cracks are entirely governed by the direction of maximal strain energy density and not by the oft-assumed principle of local symmetry. This fundamentally important result may potentially be utilized to either direct or guide running cracks.\\

\noindent\textbf{References}\\
$[$1$]$ E. Bouchbinder, T. Goldman, J. Fineberg, The dynamics of rapid fracture: instabilities, nonlinearities and length scales. Reports on Progress in Physics. 77, 046501–046501 (2014).\\\newline
$[$2$]$ T. Goldman, A. Livne, J. Fineberg, Acquisition of Inertia by a Moving Crack. Physical Review Letters. 104(2010), doi:10.1103/PhysRevLett.104.114301.\\\newline
$[$3$]$ A. Livne, E. Bouchbinder, I. Svetlizky, J. Fineberg, The Near-Tip Fields of Fast Cracks. Science. 327, 1359–1363 (2010).
