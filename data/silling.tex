Nonlocality is an essential feature of the peridynamic model of solid mechanics, which treats all internal forces as acting through finite distances. This nonlocality, which avoids the need to evaluate partial derivatives of the deformation, helps peridynamics treat singularities such as evolving cracks within its basic field equations.\\

In this talk I will offer a perspective on the significance and effect of nonlocality in the peridynamic continuum model and other theories. Nonlocality offers a mathematical tool to treat certain physical effects such as wave dispersion and attenuation in more generality than is possible in the local theory. It enables the modeling of interesting phenomena such as solitary waves, as well as fracture and fragmentation. It provides a natural compatibility of peridynamics with nanoscale long-range forces. On the other hand, nonlocality is sometimes inconvenient in macroscale simulations, for example by creating surface effects in material properties.\\

Is nonlocality real? Is it measurable in experiments or derivable from physical principles? I will consider some actual and hypothetical experiments that help to give insight into the proper role of nonlocality in continuum mechanics and the mechanics of defects