BLAS \& LaPACK are crucial and core computation components in high performance computing and machine learning applications. Historically, hardware vendors and researchers have provided optimized math kernels for specific architectures. For this reason, the application developers have to decide which library and architecture to use and re-implement thousands of lines of code to port and optimize their codes to other architectures. Moreover, following the trend of heterogeneity, hardware manufacturers and vendors are releasing new architectures and their proprietary libraries that can harness the best possible performance for commonly linear algebra kernels. However, tuned kernels for one architecture are not portable to others. Moreover, the coexistence of different architectures in a single node made orchestration difficult. To address these challenges, we introduce MatRIS, a portable framework for BLAS \& LaPACK functionalities. MatRIS ensures a separation between linear algebra algorithms and vendor library kernels using IRIS runtime. Such abstraction at the algorithm level makes implementation completely vendor-library and architecture agnostic. MatRIS uses IRIS runtime to dynamically select the vendor-library kernel and suitable processor architecture at runtime. We demonstrate that MatRIS can fully utilize different heterogeneous systems by launching and orchestrating different vendor-library kernels without any change in the source code.\\

This research reports the following contributions:

1- Improve portability and productivity for BLAS \& LaPACK codes by separating the algorithm description (application details) from the implementation, tasks mapping (hardware features), and vendor library kernels.

2- Efficient utilization of different heterogeneous systems with a large number of computing components without changing one line of code.

3- Performance study of the MatRIS implementation on three different heterogeneous systems; one NVIDIA DGX-1 system with 1× Intel CPU and 4× NVIDIA, one node of the fastest TOP5001 supercomputer today, the ORNL’s supercomputer Frontier, with 1× AMD CPU (64 cores) and 8× AMD GPUs, and one extreme heterogeneous system with 1× AMD CPU, and 8× GPUs (4× NVIDIA GPUs + 4× AMD GPUs). 