\begin{center}
\textit{Co-Author: Alain Karma}
\end{center} 
Due to their tenfold higher storing capacities, silicon and germanium have emerged as a promising replacement for Carbon-based anodes in Li-ion batteries. However, lithiation of both materials results in large (300\%) volume expansion that results in an amorphization of their crystalline structure. The significant volume expansion causes the anode particles to change shape and drives inelastic deformation, plastic flow, and fracture within the particle. Experimental observations show that lithiation in both materials is reaction controlled; thus, an invading lithiation front is created that is atomically sharp. Crucially, the initial lithiation of crystalline Si(c-Si) highly anisotropic while it remains isotropic for crystalline Ge and amorphous Si, which has been suggested as a source of its inferior performance. Here we exploit the power of the phase-field approach to describe both the motion of phase boundaries and stress-driven fracture within a self-consistent set of equations to shed light on the failure modes of Si battery anode material. We simulate the crystalline to amorphous phase transformation of Si and Ge using a modified Allen-Cahn model with anisotropic mobility to replicate the experimental observations. The mechanical response due to lithiation-driven swelling of the anode is modeled using neo-Hookean elasticity coupled with finite J-2 plasticity. Finally, we model fracture using a variational phase-field formulation that is able to describe both nucleation and propagation of cracks without interpenetration of crack surfaces under compression. We use our framework to probe deformation and fracture of Si nanopillars due to lithiation over a wide range of yield strengths. Using 1D axisymmetric simulations, we show how the stresses generated in the nanopillar attaina maximum at a critical yield strength. Then using full 2D plane stress simulations of a nanopillar cross-section, we highlight the emergence of plastic instabilities, which augment the stresses.  Finally, we combine our results of 2D simulation with 1D stability analysis of our phase-field model and experimental observations of the critical size of nanopillars for failure to estimate the lithiation-driven yield strength of Si and Ge. The results highlight the complex interplay between fracture and plasticity in the failure of the silicon anodic components. In addition, we present the results of 2D and 3D simulations that investigate the non-trivial effects of anode geometry on mechanical stability.

