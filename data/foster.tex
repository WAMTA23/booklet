\begin{center}
\textit{Co-Author: Masoud Behzadinasab}
\end{center} 
Prediction of ductile fracture, which is a prevalent failure mode in most engineering structures, is vital to numerous industries. Despite significant advancements in fracture mechanics, ductile fracture modeling has remained a challenging task and a continuing area of research. The peridynamic theory has attracted broad interest in recent years, for its innovative approach for simulating material damage. While peridynamics has been largely utilized to simulate cracking events in brittle materials, its ability in predicting ductile failure remains unclear.

We recently conducted a rigorous investigation into the capabilities of peridynamics in simulating ductile fracture in metallic alloys. The third Sandia Fracture Challenge, as a true blind prediction challenge, was employed in the examination, where the state of the art of peridynamic modeling of ductile fracture was implemented to predict deformations and failure of an additively manufactured metal, with a complex geometry, under the dynamic tensile experiments performed by Sandia National Laboratories. Following the participation in the challenge, while our modeling approach led to qualitatively good results and a correctly predicted crack path, it underpredicted the load-carrying capacity of the structure and simulated an early fracture. Our post-experiment analysis identifies the main sources of discrepancy between the blind simulations and experiments to be (1) material instabilities associated with the finite deformation peridynamic model and (2) unreliability of a Lagrangian peridynamic framework in solving problems involving extremely large deformation and extensive damage.

To address the aforementioned issues, a bond-associated, semi-Lagrangian, constitutive correspondence, peridynamic framework is proposed, in which peridynamic material point interactions depend only on their current properties (e.g. position and stress values) in the deformed configuration, and a rate-based approach is utilized to advance the state of material. A nonlocal version of the velocity gradient is presented to determine the Cauchy stress rate, using local constitutive theories, as an intermediate quantity in computing peridynamic bond forces. A bond-associated, correspondence damage modeling is introduced by using the bond-associated internal properties, e.g. stress and strain values, to incorporate classical failure criteria within the peridynamic framework. The new theory is employed to revisit the Sandia Fracture Challenge problem. Our results indicate that the new approach significantly improves the peridynamic predictions of large deformation and ductile fracture.