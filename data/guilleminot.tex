This talk is focused on stochastic modeling in fracture mechanics for quasi-brittle, heterogeneous materials. The macroscopic behavior of such materials can be strongly affected by subscale variability, including the effects of both microstructural randomness and spatial variations inelastic and fracture properties. In this talk, we present recent advances related to the modeling of spatial, multiscale-informed variability in elastic and fracture properties, and its impact on the macroscopic response [1, 2]. We also discuss how fine scale regularity can be leveraged to achieve dimensionality reduction and augment datasets obtained from simulations or physical experiments [3].\\

\noindent\textbf{References}\\
$[$1$]$ D.-A Hun, J. Guilleminot, J. Yvonnet, and M. Bornert, Stochastic Multiscale Modeling of Crack Propagation in Random Heterogeneous Media, International Journal for Numerical Methods in Engineering, 119, 1325–1344, 2019. \\\newline
$[$2$]$ T. Hu, J. Guilleminot, and J. Dolbow, A Phase-Field Model of Fracture with Traction-Free Crack Surfaces: Application to Soil Dessication and Thin-Film Fracture, submitted, 2019. \\\newline
$[$3$]$ J. Guilleminot and J. Dolbow, Data-Driven Enhancement of Fracture Paths in Random Composites, Mechanics Research Communications, Available online 16 November, 2019.