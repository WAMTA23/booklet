Numerical simulations, whether based on finite element, boundary element, peridynamics or other methods, have always been intimately linked with companion experiments. Traditionally experiments have been used to provide direct input to numerical simulations for quantities that could not be directly derived from first principles, e.g., elasto-plastic mechanical properties and the like. However, linking simulations with experiments also allowed for indirect extraction of properties that might not have been directly measurable, e.g., strength properties or thermal dependence of properties. In such cases,inverse problem techniques have been used which provide indirect extraction of the desired quantities by minimizing/optimizing some error or objective function. Finally, even for the case of fully-predictive simulations, companion experiments are critical in validating the simulation framework and its assumptions. Early attempts to couple experiments and simulations in these ways were based on the limited data obtained by point measurements (e.g., strain gauges) and were applied to simple or idealized environments (e.g., under slow loading rates and/or at room temperature). However,as full-field optical measurement techniques gained popularity, experimental data of larger portions of a structure/material allowed for much larger amounts of data to be used in these roles(i.e., material input, property extraction, or simulation validation). More recently, with the advent of the optical metrology technique of two-dimensional(2D) Digital Image Correlation (DIC), and now its three-dimensional (3D) extension Digital Volume Correlation (DVC), we can obtain very large data sets of experimental measurements of displacement and strain on the surface or in the interior of an object. In addition, this is increasingly done under extreme conditions involving high loading rate and/or temperature. In this work we will present a series of experimental measurement techniques and approaches both for the inverse numerical extraction of material properties from experiments and the validation of predictive simulations based primarily on the finite element framework. In the first we will present examples on the extraction of cohesive failure laws from both 2D and 3D experimental data. We have employed 2D DIC on materials with complex microstructures, such as Functionally Graded Materials (FGMs) and particulate reinforced composites, and have used gradient-based inverse schemes to extract cohesive failure relations for these materials. We have also employed stereo-vision DIC for the validation of thermomechanical loading simulation results. The validation methodology uses image decomposition techniques together with an error analysis framework to provide confidence levels of the simulations when compared to companion experiments.Examples will be provided for both dynamicloading situations (e.g., dynamic fracture or vibratory loading) and high temperature loading (e.g., thermomechanical fatigue).
