\begin{center}
\textit{Co-Authors: R. Vargas, A. Tsitova, F. Bernachy-Barbe, B. Bary, and R.B. Canto}
\end{center} 
In this presentation, an approach to defining the path of a curved crack in a single edge notched specimen with gray level residuals extracted from digital image correlation [1], is followed by the calibration of the parameters of a cohesive zone model [2, 3]. Only the experimental force is used in the cost function minimized in finite element model updating. The displacement and gray level residual fields allow for the validation of the calibrated parameters. Last, a phase field model is probed with the previously calibrated parameters.\\

\noindent\textbf{References}\\
$[$1$]$ F. Hild, A. Bouterf, and S. Roux, Damage Measurements via DIC,International Journal of Fracture,191(1-2):77–105, 2015.\\\newline
$[$2$]$ K. Park, G. H. Paulino, and J. R. Roesler, A unified potential-based cohesive model ofmixed-mode fracture,Journal of the Mechanics and Physics of Solids,57(6):891–908, 2009.\\\newline
$[$3$]$ K. Park and G. H. Paulino, Computational implementation of the PPR potential-based cohesive model in ABAQUS: educational perspective,Engineering Fracture Mechanics,93:239–262, 2012.
