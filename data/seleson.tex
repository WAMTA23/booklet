\begin{center}
\textit{Co-Authors: Marco Pasetto and Yohan John}
\end{center} 
Peridynamics is a nonlocal reformulation of classical continuum mechanics suitable for material failure and damage simulation. Governing equations in peridynamics are based on spatial integration rather than spatial differentiation, allowing natural representation of material dis-continuities, such as cracks. A meshfree approach proposed in [1] has been demonstrated to be an effective discretization method for large-scale engineering simulations, particularly those involving large deformation and complex fractures. However, a robust quantitative assessment of the performance of this meshfree method, particularly in fracture scenarios, is lacking. In [2], the convergence of meshfree numerical solutions of static peridynamic problems has been investigated. Related convergence studies for peridynamic wave propagation problems appeared in [3].  In this talk, we will discuss recent convergence studies of wave propagation and extensions to dynamic crack propagation in meshfree peridynamic simulations.\\

\noindent\textbf{References}\\
$[$1$]$ S. A. Silling and E. Askari, A meshfree method based on the peridynamic model of solid mechanics, Computers \& Structures, 83, 1526–1535, 2005.\\\newline
$[$2$]$ P. Seleson and D. J. Littlewood, Convergence studies in meshfree peridynamic simulations, Computers and Mathematics with Applications, 71, 2432–2448, 2016.\\\newline
$[$3$]$ P. Seleson and D. J. Littlewood, Numerical tools for improved convergence of meshfree peridynamic discretizations, in Handbook of Nonlocal Continuum Mechanics for Materials and Structures, G. Voyiadjis (ed.), Springer, Cham, 2018.
