\begin{center}
\textit{Co-Authors: Georgios Moutsanidis and Yuri Bazilevs}
\end{center} 
We introduce a phase field fracture formulation that results in a hyperbolic partial differential
equation (PDE) for the order parameter distinguishing broken and unbroken phases of the
material [1, 2]. This PDE can be formally derived from microforce balance theory by associating
a microscopic kinetic energy with the order parameter’s rate of change. An important practical
advantage is that hyperbolic PDEs are more amenable to explicit time stepping, which is almost
universally favored in simulations of extreme events like explosions, where small time steps
are needed to resolve rapid phenomena. Finite element computations of benchmark problems
demonstrate that the properties of the new model are similar to elliptic phase field models
(relative to typical modeling errors of current methods), although there is a noticeable rate-
toughening effect, which becomes more pronounced as the wave speed is reduced. The model is
then implemented into a hybrid meshfree–isogeometric hydrocode to simulate more complicated
scenarios and perform qualitative comparisons with experimental results from the literature on
blast-induced fracture of PMMA. We conclude with some discussion of ongoing and future work
on efficient implementation of isogeometric and meshfree technologies, such as using automatic
code generation [3] to rapidly test new formulations.\\

\noindent\textbf{References}\\
$[$1$]$ D. Kamensky, G. Moutsanidis, and Y. Bazilevs, Hyperbolic phase field modeling of brittle
fracture: part I—theory and simulations, Journal of the Mechanics and Physics of Solids,
121, 81–98, 2018.\\\newline
$[$2$]$ G. Moutsanidis, D. Kamensky, J.S. Chen, and Y. Bazilevs, Hyperbolic phase field modeling of brittle fracture: part II—immersed IGA–RKPM coupling for air-blast–structure
interaction, Journal of the Mechanics and Physics of Solids, 121, 114–132, 2018.\\\newline
$[$3$]$ D. Kamensky, Y. Bazilevs, tIGAr: Automating isogeometric analysis with FEniCS. Computer Methods in Applied Mechanics and Engineering, 344, 477–489, 2019.
