Comparison of numerical simulations against experimental data is essential
for the validation of fracture models in order to gain confidence in their
predictability and reliability. Peridynamics and phase field approaches have recently
delivered promising results for modeling complex fracture phenomena and significant
efforts have been carried out in the past years to validate the corresponding
fracture models using available experimental data. However, on one hand, it is still
unclear whether the data obtained from current experiments is informative enough to
satisfactorily validate models in fracture mechanics. On the other hand, it would also be
interesting to combine simulation tools and experimental design to optimize
control parameters in fracture mechanics experiments.

The objectives of this workshop are to bring together experts in experimental fracture mechanics,
peridynamics, and phase field methods to discuss the state-of-the-art of experimental
measurement and computational modeling with applications in fracture mechanics, to promote
a dialogue between these communities, and to identify challenges and pathways for robust validation
of phase field and peridynamic models as well as integration of experimental and modeling efforts.

\section*{Organization committee }
\begin{itemize}
\item Patrick Diehl, Louisiana State University
\item Pablo Seleson, Oak Ridge National Lab
\item Serge Prudhomme, Ecole de Polytechnique Montreal
\end{itemize}

\section*{Scientific committee}
\begin{itemize}
\item Stuart Silling, Sandia National Lab
\item Robert Lipton, Louisiana State University
%\item John Dolbow, Duke University
\item K. Ravi-Chandar, The University of Texas at Austin
%\item Yuri Bazilevs, Brown University 
\end{itemize}
