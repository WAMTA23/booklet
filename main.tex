\documentclass[12pt]{book}

\usepackage{xcolor}

\usepackage{tocstyle}
\usetocstyle{standard}




\usepackage[a4paper,margin=3cm,innermargin=3cm]{geometry}

\usepackage[
    type={CC},
    modifier={by-nc-nd},
    version={4.0},
]{doclicense} 

\usepackage{needspace}
\usepackage{marginnote}
\usepackage{ifxetex}
\renewcommand*{\marginfont}{\sffamily\footnotesize}

\usepackage{imakeidx}
\usepackage[hidelinks]{hyperref}
\hypersetup{
pdftitle={Book of Abstracts},
pdfsubject={WFM2020},
pdfauthor={Patrick Diehl},
pdfkeywords={phase fields,peridynamics,experimental mechanics}
}
\makeindex[intoc]

\newenvironment{conf-abstract}[4][]{
  \needspace{10\baselineskip}
  \begin{center}
    { \renewcommand\textsuperscript[1]{}
      \phantomsection\addcontentsline{toc}{section}
      {\texorpdfstring{#2 (\emph{#3})}{#2 (#3)}}
    }
    {{\large\bfseries #2}\marginnote{#1}\par}
    \medskip
    {#3\par}
    \smallskip
    {\small #4\par}
  \end{center}
}{%
  \bigskip
  \hrule
  \bigskip
}

\usepackage{etoolbox}
\newcommand{\indexauthors}[1]{%
  \forcsvlist{\index}{#1}
}

\setcounter{tocdepth}{3}
\setcounter{secnumdepth}{-1}
\pagestyle{plain}


\title{USACM thematic Workshop on Experimental and Computational Fracture Mechanics: \\
	\large Validating peridynamics and phase field models for fracture prediction and experimental design}
\author{Venue: \\ Center of Computation \& Technology \\ Louisiana State University}

\ifxetex
\usepackage{fontspec}
\setmainfont{Raleway}
\fi

\usepackage{timetable/calendar} 

\usepackage{rotating}

\usepackage{datatool}

\DTLloaddb{data}{./data.csv}
\DTLloaddb{dataPoster}{./dataPoster.csv}


\newcommand*{\thevalue}{}
\newcommand*{\getCol}[2]{%
 \DTLgetvalueforkey{\thevalue}{Last}{data}{Talk}{#2}%
}

\newcommand*{\lastname}[2]{
\getCol{1}{#1}
\thevalue 
}


\newcommand*{\thevalues}{}
\newcommand*{\getCols}[2]{%
 \DTLgetvalueforkey{\thevalues}{Time}{data}{Talk}{#2}%
}



\newcommand*{\gettime}[2]{
\getCols{1}{#1}
\kern-1ex\thevalues
}



\begin{document}

\frontmatter

\maketitle

This workshop is sponsored by

\begin{itemize}
\item Technical Thrust Area on Large Scale Structural Systems and Optimal Design of the US Association for Computational Mechanics
\item Center of Computation \& Technology at Louisiana State University
\item Oak Ridge National Laboratory
\item Society for Experimental Mechanics
\end{itemize}

\newpage

\section*{Abstract}
The comparison against experimental results is essential to validate models and discretizations in order to gain confidence in the approach’s predictability and reliability. Peridynamics and phase field methods have been utilized for fitting or validation against experimental results. When using experimental results. the question arises if the experimental data is mature enough to provide all data to set up the simulations, e.g. applied loading, and compare the experimentally measured quantity of interest with the one obtained by simulation. \\

During this workshop the phase field and peridyanmic community showcases which kind of experiments were used for validation and the difficulties phased. The experimental fracture mechanics community emphasizes the experiments they are researching on and what kind of simulations would be interesting for them to gain more understanding of experimental phenomena observed. \\

This workshop brings together experts on experimental fracture mechanics and experts in modelling and simulating crack and fractures utilizing peridyanmic and phase fields methods. As a results, it will elaborate the collaboration between the three participating communities and starts the discussion about a set of experiments used as a benchmark problem for a robust validation of phase field and peridyanmic models.

\section*{Organization committee }
\begin{itemize}
\item Patrick Diehl, Louisiana State University
\item Pablo Seleson, Oak Ridge National Lab
\item Serge Prudhomme, Ecole de Polytechnique Montreal
\end{itemize}

\section*{Scientific committee}
\begin{itemize}
\item Stuart Silling, Sandia National Lab
\item Robert Lipton, Louisiana State University
\item John Dolbow, Duke University
\item K. Ravi-Chandar, The University of Texas at Austin
\item Yuri Bazilevs, Brown University 
\end{itemize}


\chapter{Welcome Address}


Yo, welcome! Have a good time you guys!


\begin{sidewaysfigure}
\begin{calendar}{\textwidth} % Calendar to be the entire width of the page

\setcounter{calendardate}{23} % Day on which the calendar starts - note that you have to account for blank days

%----------------------------------------------------------------------------------------
%	Monday
%----------------------------------------------------------------------------------------

\day{}{}

%----------------------------------------------------------------------------------------
%	SECOND DAY
%----------------------------------------------------------------------------------------

\day{}{}

%----------------------------------------------------------------------------------------
%	THIRD DAY
%----------------------------------------------------------------------------------------

\day{Arrival}{
\textbf{Reception} The Cook Hotel (7:00-9:00) \\\daysep
}
%----------------------------------------------------------------------------------------
%	FOURTH DAY
%----------------------------------------------------------------------------------------


\day{Day 1}{
\textbf{Registration/Coffee} (8:15-8:45) \\\daysep
Welcoming Remarks (8:45-9:00)\\\daysep
\lastname{1}~ (9:00-10:00) \\\daysep
\textbf{Coffee Break} (10:00-10:30) \\\daysep
\lastname{2}~ (10:30-11:00) \\\daysep
\lastname{3}~ (11:00-11:30) \\\daysep
\lastname{4}~ (11:30-12:00) \\\daysep
\lastname{5}~ (12:00-12:30) \\\daysep
\textbf{Lunch} (12:30-14:00) \\\daysep
\lastname{6}~ (14:00-14:30) \\\daysep
\lastname{7}~ (14:30-15:00) \\\daysep
\lastname{8}~ (15:00-15:30) \\\daysep
\textit{Discussion }  (15:30-16:00) \\\daysep
\textbf{Coffee Break} (16:00-16:30) \\\daysep
\lastname{9}~ (16:30-17:00) \\\daysep
\lastname{10}~ (17:00-17:30) \\\daysep
}

%----------------------------------------------------------------------------------------
%	FIFTH DAY
%----------------------------------------------------------------------------------------

\day{Day 2}{
\textbf{Registration/Coffee} (8:30-9:00) \\\daysep
\lastname{11}~ (9:00-10:00) \\\daysep
\textbf{Coffee Break} (10:00-10:30) \\\daysep
\lastname{12}~ (10:30-11:00) \\\daysep
\lastname{13}~ (11:00-11:30) \\\daysep
\lastname{14}~ (11:30-12:00) \\\daysep
\lastname{15}~ (12:00-12:30) \\\daysep
\textbf{Lunch} (12:30-14:00) \\\daysep
\lastname{16}~ (14:00-14:30) \\\daysep
\lastname{17}~ (14:30-15:00) \\\daysep
\textit{Discussion} (15:00-15:30) \\\daysep
\textbf{Coffee Break}~ (15:30-16:00) \\\daysep
\lastname{18}~ (16:00-17:00) \\\daysep
\lastname{19}~ (17:00-17:30) \\\daysep
\textbf{Banquet} Faculty Club LSU (19:00-21:00)\\
}

%----------------------------------------------------------------------------------------
%	SIXTH DAY
%----------------------------------------------------------------------------------------


\day{Day 3}{
\textbf{Registration/Coffee} (8:30-9:00) \\\daysep
\lastname{20}~ (9:00-09:30) \\\daysep
\lastname{21}~ (9:30-10:00) \\\daysep
\textbf{Coffee Break}~ (10:00-10:30) \\\daysep
\lastname{22}~ (10:30-11:00) \\\daysep
\lastname{23}~ (11:00-11:30) \\\daysep
\lastname{24}~ (11:30-12:00) \\\daysep
General discussion (12:00--12:30) \\\daysep
\textbf{Lunch} (12:30-14:00) \\\daysep
}



%----------------------------------------------------------------------------------------
%	SEVENTH DAY
%----------------------------------------------------------------------------------------





% Note: more days can be added to give the calendar a third or fourth week

%----------------------------------------------------------------------------------------

\finishCalendar
\end{calendar}
\end{sidewaysfigure}


\tableofcontents

\mainmatter

\chapter{Industrial talk}

\begin{conf-abstract}[26$^{th}$\\???]
{INTEGRATED  COMPUTATIONAL  MATERIALS ENGINEERING  (ICME)  DEVELOPMENT  OF CARBONFIBER COMPOSITES  FOR  LIGHTWEIGHT  VEHICLES}
{Danielle Zeng}
{Ford Motor Company}
\indexauthors{Zeng!Danielle}
\begin{center}
\textit{Co-Authors: Pablo Seleson, Bo Ren, and C.T. Wu}
\end{center}
Automotive   manufacturers  use  lightweight   materials  to  meet  the  increasing   demands  of  fue l efficiency.  The Carbon Fiber  Reinforced  Polymer  (CFRP) composites,  with  a density  of 1.55  g/cm3 and  a tensile  strength  of 2000  MPa in  the fiber  direction,  are among  the most  promising  candidate s  to  replace  the  metals  currently  used  for  structural  components.  It  is  important  to  note  that  the  performance  of  carbon  fiber  composites  is  determined  not  only  by  the  component  design,  but  also the  manufacturing   processes.  In  this   talk,   the  focus  is  on   the  application   of  an  Integrated Computational  Materials  Engineering  (ICME) approach  to  the structural  composite  design.  A suite  of  predictive   models   is  developed   to  link   materials   design,   manufacturing   process  and  fina l performance  to  enable  optimal   design  and  manufacturing   of  CFRP  components  for  automotive  vehicles.

One of the greatest challenges  for successful applying  the ICME approach  to CF composites  is  how  to  accurately simulate  the different  failure  modes  during  crash scenarios.  Especially,  the traditiona l thin  shell  model  in  finite  element  simulation  has  difficulty  in  capturing  the  delamination  behavior during  complex  loading  conditions.  Recently,  a discontinuous  Galerkin  weak form  for bond-based peridynamic  models  is developed  for composite  modeling  through  the collaboration  among  ORNL, LSTC  and   Ford.     The  accuracy  and   computational   efficiency   of  the  developed   model   for delamination   modeling   is  demonstrated  through  simulating   a dynamic  bending  test  of  a  lamina te  structure.
\end{conf-abstract}


\chapter{Talks}

\DTLforeach*{data}{\last=Last,\first=First,\affiliation=Affiliation,\title=Title,\datum=Date,\time=Time,\text=Abstract}
{
\begin{conf-abstract}[\datum\\\tiny\time]
{\title}
{\first~ \last}
{\affiliation}
\indexauthors{\last!\first}
\input{\text}
\end{conf-abstract}
}

\chapter{Posters}

\DTLforeach*{dataPoster}{\last=Last,\first=First,\affiliation=Affiliation,\title=Title,\datum=Date,\time=Time,\text=Abstract}
{
\begin{conf-abstract}[\datum\\\time]
{\title}
{\first~ \last}
{\affiliation}
\indexauthors{\last!\first}
\begin{center}
\input{\text}
\end{center}
\end{conf-abstract}
}


% Specify conf-abstract like this:
% \begin{conf-abstract}[optional text going into the margin note]
% {Title of Paper}
% {Authors (use \textsuperscript as institution markers)}
% {Institutions (use \textsuperscript as institution markers)}
% \indexauthors{Lastname1!Firstname 1, Lastname2!Firstname2}
% Abstract text
% \end{conf-abstract}
%
% It's probably best to generate the abstracts from a 
% database or something via a script. Don't forget to
% check through for any special characters that need to
% be escaped.

%\input{abstracts/paper1}
%\input{abstracts/paper2}
%\input{abstracts/paper3}

\chapter{Additional information}

\section{Addresses}

\subsection*{Workshop venue}
Center for Computation and Technology, 340 E Parker Blvd, Baton Rouge, LA 70808 
\subsection*{Hotel}
The Cook Hotel at LSU, 3848 W Lakeshore Dr, Baton Rouge, LA 70808 \\
(225) 383-2665
\subsection*{Banquet}
LSU Faculty Club, 101 Tower Dr, Baton Rouge, LA 70803 \\
(225) 578-2356

\section{Restaurants}

\subsection*{Walking distance}

\begin{itemize}
\item The Chimes -- Lively campus-area hangout from a local chain featuring a worldwide beer list \& hearty bar fare: 3357 Highland Rd, Baton Rouge, LA 70802 (225 383-1754)
\item Louie's Cafe -- LSU-area fixture dating to 1941 serves a diner menu 24/7 in a classic lunch-counter setting: 3322 Lake St, Baton Rouge, LA 70802 (225 346-8221)
\item Highland Coffees -- Charming, airy locale with a laid-back vibe for coffee roasted on-site \& a variety of baked goods: 3350 Highland Rd, Baton Rouge, LA 70802 (225 336-9773)
\end{itemize}

\subsection*{Local cousin}

\begin{itemize}
\item Parrain's Seafood Restaurant -- Local seafood specialist cooking up Louisiana recipes in a rustic space with porch seating: 3225 Perkins Rd, Baton Rouge, LA 70808 (225 381-9922)
\item Mike Anderson's - Baton Rouge -- Area staple for regional seafood in a spacious, wood-lined setting with a sports-friendly vibe: 1031 W Lee Dr, Baton Rouge, LA 70820 (225 766-7823)
\item Stroubes Seafood and Steaks -- Chophouse presenting local preparations of meat \& seafood in comfortable digs with a lounge: 107 3rd St, Baton Rouge, LA 70801 (225 448-2830)
\end{itemize}


\backmatter
\renewcommand{\indexname}{Author Index}
\printindex
\newpage
\doclicenseThis 

\end{document}
